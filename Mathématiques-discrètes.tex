% Options for packages loaded elsewhere
\PassOptionsToPackage{unicode}{hyperref}
\PassOptionsToPackage{hyphens}{url}
%
\documentclass[
  letterpaper,
]{scrbook}

\usepackage{amsmath,amssymb}
\usepackage{lmodern}
\usepackage{iftex}
\ifPDFTeX
  \usepackage[T1]{fontenc}
  \usepackage[utf8]{inputenc}
  \usepackage{textcomp} % provide euro and other symbols
\else % if luatex or xetex
  \usepackage{unicode-math}
  \defaultfontfeatures{Scale=MatchLowercase}
  \defaultfontfeatures[\rmfamily]{Ligatures=TeX,Scale=1}
\fi
% Use upquote if available, for straight quotes in verbatim environments
\IfFileExists{upquote.sty}{\usepackage{upquote}}{}
\IfFileExists{microtype.sty}{% use microtype if available
  \usepackage[]{microtype}
  \UseMicrotypeSet[protrusion]{basicmath} % disable protrusion for tt fonts
}{}
\makeatletter
\@ifundefined{KOMAClassName}{% if non-KOMA class
  \IfFileExists{parskip.sty}{%
    \usepackage{parskip}
  }{% else
    \setlength{\parindent}{0pt}
    \setlength{\parskip}{6pt plus 2pt minus 1pt}}
}{% if KOMA class
  \KOMAoptions{parskip=half}}
\makeatother
\usepackage{xcolor}
\usepackage[paper=a4paper,top=2cm,bottom=2cm,left=1.5cm,right=3.5cm,twoside]{geometry}
\setlength{\emergencystretch}{3em} % prevent overfull lines
\setcounter{secnumdepth}{5}
% Make \paragraph and \subparagraph free-standing
\ifx\paragraph\undefined\else
  \let\oldparagraph\paragraph
  \renewcommand{\paragraph}[1]{\oldparagraph{#1}\mbox{}}
\fi
\ifx\subparagraph\undefined\else
  \let\oldsubparagraph\subparagraph
  \renewcommand{\subparagraph}[1]{\oldsubparagraph{#1}\mbox{}}
\fi

\usepackage{color}
\usepackage{fancyvrb}
\newcommand{\VerbBar}{|}
\newcommand{\VERB}{\Verb[commandchars=\\\{\}]}
\DefineVerbatimEnvironment{Highlighting}{Verbatim}{commandchars=\\\{\}}
% Add ',fontsize=\small' for more characters per line
\newenvironment{Shaded}{}{}
\newcommand{\AlertTok}[1]{\textcolor[rgb]{1.00,0.00,0.00}{\textbf{#1}}}
\newcommand{\AnnotationTok}[1]{\textcolor[rgb]{0.38,0.63,0.69}{\textbf{\textit{#1}}}}
\newcommand{\AttributeTok}[1]{\textcolor[rgb]{0.49,0.56,0.16}{#1}}
\newcommand{\BaseNTok}[1]{\textcolor[rgb]{0.25,0.63,0.44}{#1}}
\newcommand{\BuiltInTok}[1]{#1}
\newcommand{\CharTok}[1]{\textcolor[rgb]{0.25,0.44,0.63}{#1}}
\newcommand{\CommentTok}[1]{\textcolor[rgb]{0.38,0.63,0.69}{\textit{#1}}}
\newcommand{\CommentVarTok}[1]{\textcolor[rgb]{0.38,0.63,0.69}{\textbf{\textit{#1}}}}
\newcommand{\ConstantTok}[1]{\textcolor[rgb]{0.53,0.00,0.00}{#1}}
\newcommand{\ControlFlowTok}[1]{\textcolor[rgb]{0.00,0.44,0.13}{\textbf{#1}}}
\newcommand{\DataTypeTok}[1]{\textcolor[rgb]{0.56,0.13,0.00}{#1}}
\newcommand{\DecValTok}[1]{\textcolor[rgb]{0.25,0.63,0.44}{#1}}
\newcommand{\DocumentationTok}[1]{\textcolor[rgb]{0.73,0.13,0.13}{\textit{#1}}}
\newcommand{\ErrorTok}[1]{\textcolor[rgb]{1.00,0.00,0.00}{\textbf{#1}}}
\newcommand{\ExtensionTok}[1]{#1}
\newcommand{\FloatTok}[1]{\textcolor[rgb]{0.25,0.63,0.44}{#1}}
\newcommand{\FunctionTok}[1]{\textcolor[rgb]{0.02,0.16,0.49}{#1}}
\newcommand{\ImportTok}[1]{#1}
\newcommand{\InformationTok}[1]{\textcolor[rgb]{0.38,0.63,0.69}{\textbf{\textit{#1}}}}
\newcommand{\KeywordTok}[1]{\textcolor[rgb]{0.00,0.44,0.13}{\textbf{#1}}}
\newcommand{\NormalTok}[1]{#1}
\newcommand{\OperatorTok}[1]{\textcolor[rgb]{0.40,0.40,0.40}{#1}}
\newcommand{\OtherTok}[1]{\textcolor[rgb]{0.00,0.44,0.13}{#1}}
\newcommand{\PreprocessorTok}[1]{\textcolor[rgb]{0.74,0.48,0.00}{#1}}
\newcommand{\RegionMarkerTok}[1]{#1}
\newcommand{\SpecialCharTok}[1]{\textcolor[rgb]{0.25,0.44,0.63}{#1}}
\newcommand{\SpecialStringTok}[1]{\textcolor[rgb]{0.73,0.40,0.53}{#1}}
\newcommand{\StringTok}[1]{\textcolor[rgb]{0.25,0.44,0.63}{#1}}
\newcommand{\VariableTok}[1]{\textcolor[rgb]{0.10,0.09,0.49}{#1}}
\newcommand{\VerbatimStringTok}[1]{\textcolor[rgb]{0.25,0.44,0.63}{#1}}
\newcommand{\WarningTok}[1]{\textcolor[rgb]{0.38,0.63,0.69}{\textbf{\textit{#1}}}}

\providecommand{\tightlist}{%
  \setlength{\itemsep}{0pt}\setlength{\parskip}{0pt}}\usepackage{longtable,booktabs,array}
\usepackage{calc} % for calculating minipage widths
% Correct order of tables after \paragraph or \subparagraph
\usepackage{etoolbox}
\makeatletter
\patchcmd\longtable{\par}{\if@noskipsec\mbox{}\fi\par}{}{}
\makeatother
% Allow footnotes in longtable head/foot
\IfFileExists{footnotehyper.sty}{\usepackage{footnotehyper}}{\usepackage{footnote}}
\makesavenoteenv{longtable}
\usepackage{graphicx}
\makeatletter
\def\maxwidth{\ifdim\Gin@nat@width>\linewidth\linewidth\else\Gin@nat@width\fi}
\def\maxheight{\ifdim\Gin@nat@height>\textheight\textheight\else\Gin@nat@height\fi}
\makeatother
% Scale images if necessary, so that they will not overflow the page
% margins by default, and it is still possible to overwrite the defaults
% using explicit options in \includegraphics[width, height, ...]{}
\setkeys{Gin}{width=\maxwidth,height=\maxheight,keepaspectratio}
% Set default figure placement to htbp
\makeatletter
\def\fps@figure{htbp}
\makeatother
\newlength{\cslhangindent}
\setlength{\cslhangindent}{1.5em}
\newlength{\csllabelwidth}
\setlength{\csllabelwidth}{3em}
\newlength{\cslentryspacingunit} % times entry-spacing
\setlength{\cslentryspacingunit}{\parskip}
\newenvironment{CSLReferences}[2] % #1 hanging-ident, #2 entry spacing
 {% don't indent paragraphs
  \setlength{\parindent}{0pt}
  % turn on hanging indent if param 1 is 1
  \ifodd #1
  \let\oldpar\par
  \def\par{\hangindent=\cslhangindent\oldpar}
  \fi
  % set entry spacing
  \setlength{\parskip}{#2\cslentryspacingunit}
 }%
 {}
\usepackage{calc}
\newcommand{\CSLBlock}[1]{#1\hfill\break}
\newcommand{\CSLLeftMargin}[1]{\parbox[t]{\csllabelwidth}{#1}}
\newcommand{\CSLRightInline}[1]{\parbox[t]{\linewidth - \csllabelwidth}{#1}\break}
\newcommand{\CSLIndent}[1]{\hspace{\cslhangindent}#1}

\makeatletter
\@ifpackageloaded{tcolorbox}{}{\usepackage[many]{tcolorbox}}
\@ifpackageloaded{fontawesome5}{}{\usepackage{fontawesome5}}
\definecolor{quarto-callout-color}{HTML}{909090}
\definecolor{quarto-callout-note-color}{HTML}{0758E5}
\definecolor{quarto-callout-important-color}{HTML}{CC1914}
\definecolor{quarto-callout-warning-color}{HTML}{EB9113}
\definecolor{quarto-callout-tip-color}{HTML}{00A047}
\definecolor{quarto-callout-caution-color}{HTML}{FC5300}
\definecolor{quarto-callout-color-frame}{HTML}{acacac}
\definecolor{quarto-callout-note-color-frame}{HTML}{4582ec}
\definecolor{quarto-callout-important-color-frame}{HTML}{d9534f}
\definecolor{quarto-callout-warning-color-frame}{HTML}{f0ad4e}
\definecolor{quarto-callout-tip-color-frame}{HTML}{02b875}
\definecolor{quarto-callout-caution-color-frame}{HTML}{fd7e14}
\makeatother
\makeatletter
\makeatother
\makeatletter
\@ifpackageloaded{bookmark}{}{\usepackage{bookmark}}
\makeatother
\makeatletter
\@ifpackageloaded{caption}{}{\usepackage{caption}}
\AtBeginDocument{%
\ifdefined\contentsname
  \renewcommand*\contentsname{Table des matières}
\else
  \newcommand\contentsname{Table des matières}
\fi
\ifdefined\listfigurename
  \renewcommand*\listfigurename{Liste des Figures}
\else
  \newcommand\listfigurename{Liste des Figures}
\fi
\ifdefined\listtablename
  \renewcommand*\listtablename{Liste des Tables}
\else
  \newcommand\listtablename{Liste des Tables}
\fi
\ifdefined\figurename
  \renewcommand*\figurename{Figure}
\else
  \newcommand\figurename{Figure}
\fi
\ifdefined\tablename
  \renewcommand*\tablename{Table}
\else
  \newcommand\tablename{Table}
\fi
}
\@ifpackageloaded{float}{}{\usepackage{float}}
\floatstyle{ruled}
\@ifundefined{c@chapter}{\newfloat{codelisting}{h}{lop}}{\newfloat{codelisting}{h}{lop}[chapter]}
\floatname{codelisting}{Listing}
\newcommand*\listoflistings{\listof{codelisting}{Liste des Listings}}
\usepackage{amsthm}
\theoremstyle{definition}
\newtheorem{definition}{Définition}[chapter]
\theoremstyle{definition}
\newtheorem{example}{Exemple}[chapter]
\theoremstyle{remark}
\renewcommand*{\proofname}{Preuve}
\newtheorem*{remark}{Remarque}
\newtheorem*{solution}{Solution}
\makeatother
\makeatletter
\@ifpackageloaded{caption}{}{\usepackage{caption}}
\@ifpackageloaded{subcaption}{}{\usepackage{subcaption}}
\makeatother
\makeatletter
\@ifpackageloaded{tcolorbox}{}{\usepackage[many]{tcolorbox}}
\makeatother
\makeatletter
\@ifundefined{shadecolor}{\definecolor{shadecolor}{HTML}{d5d6db}}
\makeatother
\makeatletter
\makeatother
\ifLuaTeX
  \usepackage{selnolig}  % disable illegal ligatures
\fi
\IfFileExists{bookmark.sty}{\usepackage{bookmark}}{\usepackage{hyperref}}
\IfFileExists{xurl.sty}{\usepackage{xurl}}{} % add URL line breaks if available
\urlstyle{same} % disable monospaced font for URLs
\hypersetup{
  pdftitle={Mathématiques discrètes},
  pdfauthor={Marc-André Désautels},
  hidelinks,
  pdfcreator={LaTeX via pandoc}}

\title{Mathématiques discrètes}
\author{Marc-André Désautels}
\date{11/27/22}

\begin{document}
\frontmatter
\maketitle
\ifdefined\Shaded\renewenvironment{Shaded}{\begin{tcolorbox}[frame hidden, colback={shadecolor}, enhanced, breakable, boxrule=0pt]}{\end{tcolorbox}}\fi

\renewcommand*\contentsname{Table des matières}
{
\setcounter{tocdepth}{2}
\tableofcontents
}
\mainmatter
\bookmarksetup{startatroot}

\hypertarget{pruxe9face}{%
\chapter*{Préface}\label{pruxe9face}}
\addcontentsline{toc}{chapter}{Préface}

\markboth{Préface}{Préface}

Ce document est un livre Quarto.

Pour en apprendre davantage sur les livres Quarto, visitez
\url{https://quarto.org/docs/books}.

\bookmarksetup{startatroot}

\hypertarget{systuxe8mes-de-numuxe9ration}{%
\chapter{Systèmes de numération}\label{systuxe8mes-de-numuxe9ration}}

\hypertarget{introduction}{%
\section{Introduction}\label{introduction}}

\hypertarget{systuxe8mes-positionnels}{%
\section{Systèmes positionnels}\label{systuxe8mes-positionnels}}

\hypertarget{division-entiuxe8re}{%
\section{Division entière}\label{division-entiuxe8re}}

\hypertarget{conversions}{%
\section{Conversions}\label{conversions}}

\hypertarget{vers-la-base-10}{%
\subsection{Vers la base 10}\label{vers-la-base-10}}

\hypertarget{de-la-base-10-vers-une-autre-base}{%
\subsection{De la base 10 vers une autre
base}\label{de-la-base-10-vers-une-autre-base}}

\hypertarget{opuxe9rations-en-base-2}{%
\section{Opérations en base 2}\label{opuxe9rations-en-base-2}}

\hypertarget{arithmuxe9tique-de-lordinateur}{%
\section{Arithmétique de
l'ordinateur}\label{arithmuxe9tique-de-lordinateur}}

\hypertarget{repruxe9sentation-des-entiers}{%
\subsection{Représentation des
entiers}\label{repruxe9sentation-des-entiers}}

\hypertarget{repruxe9sentation-des-nombres-ruxe9els}{%
\subsection{Représentation des nombres
réels}\label{repruxe9sentation-des-nombres-ruxe9els}}

\hypertarget{la-norme-ieee754}{%
\section{La norme IEEE754}\label{la-norme-ieee754}}

\bookmarksetup{startatroot}

\hypertarget{arithmuxe9tique-des-nombres-en-base-n}{%
\chapter{\texorpdfstring{Arithmétique des nombres en base
\(n\)}{Arithmétique des nombres en base n}}\label{arithmuxe9tique-des-nombres-en-base-n}}

\hypertarget{nombres-en-base-n}{%
\section{\texorpdfstring{Nombres en base
\(n\)}{Nombres en base n}}\label{nombres-en-base-n}}

Un système de numération est un ensemble de règles qui permettent de
représenter des nombres. Le plus ancien est probablement le système
unaire où le symbole \textbar{} représente l'entier un,
\textbar\textbar{} représente l'entier deux, \textbar\textbar\textbar{}
pour trois, \textbar\textbar\textbar\textbar{} pour quatre et ainsi de
suite. Ce système atteint vite ses limites, mais il permet de mettre en
évidence le fait qu'il existe plusieurs façons de représenter les
entiers.

\begin{longtable}[]{@{}
  >{\centering\arraybackslash}p{(\columnwidth - 6\tabcolsep) * \real{0.2022}}
  >{\centering\arraybackslash}p{(\columnwidth - 6\tabcolsep) * \real{0.3146}}
  >{\centering\arraybackslash}p{(\columnwidth - 6\tabcolsep) * \real{0.2360}}
  >{\centering\arraybackslash}p{(\columnwidth - 6\tabcolsep) * \real{0.2472}}@{}}
\toprule()
\begin{minipage}[b]{\linewidth}\centering
\textbf{Nom français}
\end{minipage} & \begin{minipage}[b]{\linewidth}\centering
\textbf{Système unaire}
\end{minipage} & \begin{minipage}[b]{\linewidth}\centering
\textbf{Système décimal}
\end{minipage} & \begin{minipage}[b]{\linewidth}\centering
\textbf{Chiffres romains}
\end{minipage} \\
\midrule()
\endhead
Zéro & & 0 & \\
Un & \textbar{} & 1 & I \\
Deux & \textbar\textbar{} & 2 & II \\
Trois & \textbar\textbar\textbar{} & 3 & III \\
\(\vdots\) & \(\vdots\) & \(\vdots\) & \(\vdots\) \\
Douze & \textbar\textbar\textbar\textbar{}
\textbar\textbar\textbar\textbar{} \textbar\textbar\textbar\textbar{} &
12 & XII \\
\(\vdots\) & \(\vdots\) & \(\vdots\) & \(\vdots\) \\
\bottomrule()
\end{longtable}

Dans la table ci-dessus, on remarque que sur une ligne donnée, on
retrouve quatre manières différentes de représenter le même entier. Pour
le reste de cette section, il sera important de dissocier la
\textbf{représentation} d'un nombre et sa \textbf{valeur}.

\leavevmode\vadjust pre{\hypertarget{def-systeme-numeration}{}}%
\begin{definition}[Système de numération]\label{def-systeme-numeration}

Un \textbf{système de numération} permet de compter des objets et de les
représenter par des nombres. Un système de numération
\textbf{positionnel} possède trois éléments:

\begin{itemize}
\tightlist
\item
  Base \(b\) (un entier supérieur à 1)
\item
  Symboles (digits): 0, 1, 2, \ldots, \(b\)-1
\item
  Poids des symboles selon la position et la base, où
  poids=base\textsuperscript{position}
\end{itemize}

\end{definition}

\begin{tcolorbox}[enhanced jigsaw, arc=.35mm, breakable, rightrule=.15mm, left=2mm, colbacktitle=quarto-callout-note-color!10!white, colframe=quarto-callout-note-color-frame, coltitle=black, titlerule=0mm, leftrule=.75mm, toprule=.15mm, bottomtitle=1mm, opacityback=0, title=\textcolor{quarto-callout-note-color}{\faInfo}\hspace{0.5em}{Note}, toptitle=1mm, bottomrule=.15mm, opacitybacktitle=0.6, colback=white]

Lorsque plusieurs bases interviennent dans un même contexte, on écrit
\((a_n \ldots a_1a_0)_b\) pour indiquer que le nombre représenté en base
\(b\).

\end{tcolorbox}

Le tableau ci-dessous liste les bases les plus fréquemment utilisées en
informatique.

\begin{longtable}[]{@{}
  >{\raggedright\arraybackslash}p{(\columnwidth - 4\tabcolsep) * \real{0.1831}}
  >{\centering\arraybackslash}p{(\columnwidth - 4\tabcolsep) * \real{0.1408}}
  >{\raggedright\arraybackslash}p{(\columnwidth - 4\tabcolsep) * \real{0.6761}}@{}}
\toprule()
\begin{minipage}[b]{\linewidth}\raggedright
\textbf{Nom}
\end{minipage} & \begin{minipage}[b]{\linewidth}\centering
\textbf{Base}
\end{minipage} & \begin{minipage}[b]{\linewidth}\raggedright
\textbf{Chiffres}
\end{minipage} \\
\midrule()
\endhead
Binaire & 2 & 0, 1 \\
Octal & 8 & 0, 1, 2, 3, 4, 5, 6, 7 \\
Décimal & 10 & 0, 1, 2, 3, 4, 5, 6, 7, 8, 9 \\
Hexadécimal & 16 & 0, 1, 2, 3, 4, 5, 6, 7, 8, 9, A, B, C, D, E, F \\
\bottomrule()
\end{longtable}

On remarque qu'en base 16, les dix chiffres de 0 à 9 ne suffisent pas.
Il faut donc se doter de 6 symboles additionnels. On utilise les lettres
de A à F avec la signification suivante:

\[
(A)_{16}=(10)_{10}, \quad (B)_{16}=(11)_{10}, \quad (C)_{16}=(12)_{10}, \quad (D)_{16}=(13)_{10}, \quad (E)_{16}=(14)_{10}, \quad (F)_{16}=(15)_{10}
\]

\hypertarget{le-systuxe8me-de-numuxe9ration-duxe9cimal}{%
\subsection{Le système de numération
décimal}\label{le-systuxe8me-de-numuxe9ration-duxe9cimal}}

Il s'agit du système de numération le plus utilisé dans notre société.
On peut le résumer avec les trois règles suivantes.

\begin{itemize}
\tightlist
\item
  Base = 10
\item
  Symboles ordonnés qu'on nomme les \emph{chiffres} : 0, 1, 2, 3, 4, 5,
  6, 7, 8, 9.
\item
  Le poids des symboles est donné par 10\textsuperscript{position}
\end{itemize}

\leavevmode\vadjust pre{\hypertarget{exm-decimal-3482}{}}%
\begin{example}[]\label{exm-decimal-3482}

Représentez le nombre 3482 sous une forme de numération positionnelle.

\begin{longtable}[]{@{}
  >{\raggedright\arraybackslash}p{(\columnwidth - 8\tabcolsep) * \real{0.3939}}
  >{\centering\arraybackslash}p{(\columnwidth - 8\tabcolsep) * \real{0.1515}}
  >{\centering\arraybackslash}p{(\columnwidth - 8\tabcolsep) * \real{0.1515}}
  >{\centering\arraybackslash}p{(\columnwidth - 8\tabcolsep) * \real{0.1515}}
  >{\centering\arraybackslash}p{(\columnwidth - 8\tabcolsep) * \real{0.1515}}@{}}
\toprule()
\begin{minipage}[b]{\linewidth}\raggedright
\textbf{Symboles (digits)}
\end{minipage} & \begin{minipage}[b]{\linewidth}\centering
3
\end{minipage} & \begin{minipage}[b]{\linewidth}\centering
4
\end{minipage} & \begin{minipage}[b]{\linewidth}\centering
8
\end{minipage} & \begin{minipage}[b]{\linewidth}\centering
2
\end{minipage} \\
\midrule()
\endhead
\textbf{Rang (position)} & \(\phantom{V}\) & \(\phantom{V}\) &
\(\phantom{V}\) & \(\phantom{V}\) \\
\textbf{Poids} & & & & \\
\textbf{Valeur du poids} & & & & \\
\textbf{Valeur de chaque symbole (digits)} & & & & \\
\bottomrule()
\end{longtable}

Nous avons donc que 3482=

\end{example}

\leavevmode\vadjust pre{\hypertarget{def-representation-polynomiale}{}}%
\begin{definition}[Représentation
polynomiale]\label{def-representation-polynomiale}

Le système positionnel utilise la \textbf{représentation polynomiale}.
Celle-ci est donnée par: \[
(a_na_{n-1}\ldots a_1a_0,a_{-1}a_{-2}\ldots a_{-m})_b = a_nb^n+a_{n-1}b^{n-1}+\ldots +a_1b^1+a_0b^0+a_{-1}b^{-1}+\ldots +a_{-m}b^{-m}
\] où \(b\) est la \textbf{base} et les \(a_i\) sont des
\textbf{coefficients} (les symboles de votre système de numération).

\end{definition}

\begin{tcolorbox}[enhanced jigsaw, arc=.35mm, breakable, rightrule=.15mm, left=2mm, colbacktitle=quarto-callout-important-color!10!white, colframe=quarto-callout-important-color-frame, coltitle=black, titlerule=0mm, leftrule=.75mm, toprule=.15mm, bottomtitle=1mm, opacityback=0, title=\textcolor{quarto-callout-important-color}{\faExclamation}\hspace{0.5em}{Important}, toptitle=1mm, bottomrule=.15mm, opacitybacktitle=0.6, colback=white]

Pour convertir un nombre de la base \(b\) vers la base 10 (décimal), on
trouve sa représentation polynomiale.

\end{tcolorbox}

\leavevmode\vadjust pre{\hypertarget{exm-octal-vers-decimal}{}}%
\begin{example}[]\label{exm-octal-vers-decimal}

En utilisant la représentation polynomale en base 10, convertissez le
nombre (176,21)\textsubscript{8}.

\end{example}

\hypertarget{les-nombres-binaires}{%
\subsection{Les nombres binaires}\label{les-nombres-binaires}}

Ce concept est essentiel en informatique, puisque les processeurs des
ordinateurs sont composés de transistors ne gérant que deux états chacun
(0 ou 1). Un calcul informatique n'est donc qu'une suite d'opérations
sur des paquets de 0 et de 1, appelés \textbf{bits}.

\begin{itemize}
\tightlist
\item
  Base: 2
\item
  Symboles: 0, 1
\item
  Poids: puissances de 2
\end{itemize}

\begin{tcolorbox}[enhanced jigsaw, arc=.35mm, breakable, rightrule=.15mm, left=2mm, colbacktitle=quarto-callout-important-color!10!white, colframe=quarto-callout-important-color-frame, coltitle=black, titlerule=0mm, leftrule=.75mm, toprule=.15mm, bottomtitle=1mm, opacityback=0, title=\textcolor{quarto-callout-important-color}{\faExclamation}\hspace{0.5em}{Important}, toptitle=1mm, bottomrule=.15mm, opacitybacktitle=0.6, colback=white]

En base 2, le \emph{chiffre} 2 n'existe pas (c'est un \textbf{nombre});
tout comme le \emph{chiffre} 10 n'existe pas en base 10 (c'est un
\textbf{nombre}).

\end{tcolorbox}

\leavevmode\vadjust pre{\hypertarget{exm-11001-en-decimal}{}}%
\begin{example}[]\label{exm-11001-en-decimal}

Convertissez le nombre (11001)\textsubscript{2} en décimal.

\begin{longtable}[]{@{}
  >{\raggedright\arraybackslash}p{(\columnwidth - 8\tabcolsep) * \real{0.3939}}
  >{\centering\arraybackslash}p{(\columnwidth - 8\tabcolsep) * \real{0.1515}}
  >{\centering\arraybackslash}p{(\columnwidth - 8\tabcolsep) * \real{0.1515}}
  >{\centering\arraybackslash}p{(\columnwidth - 8\tabcolsep) * \real{0.1515}}
  >{\centering\arraybackslash}p{(\columnwidth - 8\tabcolsep) * \real{0.1515}}@{}}
\toprule()
\begin{minipage}[b]{\linewidth}\raggedright
\textbf{Symboles (digits)}
\end{minipage} & \begin{minipage}[b]{\linewidth}\centering
3
\end{minipage} & \begin{minipage}[b]{\linewidth}\centering
4
\end{minipage} & \begin{minipage}[b]{\linewidth}\centering
8
\end{minipage} & \begin{minipage}[b]{\linewidth}\centering
2
\end{minipage} \\
\midrule()
\endhead
\textbf{Rang (position)} & \(\phantom{V}\) & \(\phantom{V}\) &
\(\phantom{V}\) & \(\phantom{V}\) \\
\textbf{Poids} & & & & \\
\textbf{Valeur du poids} & & & & \\
\textbf{Valeur de chaque symbole (digits)} & & & & \\
\bottomrule()
\end{longtable}

Nous avons donc que (11001)\textsubscript{2} =

\end{example}

\leavevmode\vadjust pre{\hypertarget{exm-binaire-to-decimal}{}}%
\begin{example}[]\label{exm-binaire-to-decimal}

Convertissez les nombres suivants en base 10 (décimal).

\begin{enumerate}
\def\labelenumi{(\alph{enumi})}
\tightlist
\item
  (110)\textsubscript{2} =
\item
  (101101)\textsubscript{2} =
\item
  (0,1011)\textsubscript{2} =
\item
  (110,101)\textsubscript{2} =
\end{enumerate}

\end{example}

\leavevmode\vadjust pre{\hypertarget{exm-nombres-succedent-0-base-2-1}{}}%
\begin{example}[]\label{exm-nombres-succedent-0-base-2-1}

Quels sont les nombres qui, dans la base deux, succèdent à
(0)\textsubscript{2}?

\end{example}

\leavevmode\vadjust pre{\hypertarget{exm-nombres-succedent-0-base-2-2}{}}%
\begin{example}[]\label{exm-nombres-succedent-0-base-2-2}

Quels sont les nombres qui, dans la base deux, succèdent à
(1110)\textsubscript{2}?

\end{example}

\hypertarget{uxe9tendue-des-valeurs-pour-les-entiers}{%
\subsubsection{Étendue des valeurs pour les
entiers}\label{uxe9tendue-des-valeurs-pour-les-entiers}}

En utilisant \(n\) bits, on peut former \(2^n\) entiers différents et le
plus grand d'entre eux est égal à \(2^n-1\).

\leavevmode\vadjust pre{\hypertarget{exm-entiers-binaires}{}}%
\begin{example}[]\label{exm-entiers-binaires}

Combien d'entiers différents pouvez-vous écrire en binaire en utilisant
8 \textbf{bits}? Quel est l'entier maximal que vous pouvez représenter
avec 8 \textbf{bits}?

\end{example}

\hypertarget{vocabulaire}{%
\subsubsection{Vocabulaire}\label{vocabulaire}}

Les codes binaires sont incontournables en informatique, car
l'information la plus élémentaire est le \textbf{bit}
(\emph{binary-digit}).

\begin{description}
\tightlist
\item[\textbf{Quartet}]
Nombre binaire composé de 4 éléments binaires.
\item[\textbf{Octet} (\emph{byte})]
Nombre binaire composé de 8 éléments binaires.
\item[\textbf{Mot}]
Nombre binaire composé de 16, 32 ou 64 éléments binaires.
\item[\textbf{LSB} (Least Significant Bit)]
Bit le moins significatif ou bit de poids faible (élément le plus à
droite).
\item[\textbf{MSB} (Most Significant Bit)]
Bit le plus significatif ou bit de poids fort (élément le plus à
gauche).
\end{description}

\begin{tcolorbox}[enhanced jigsaw, arc=.35mm, breakable, rightrule=.15mm, left=2mm, colbacktitle=quarto-callout-tip-color!10!white, colframe=quarto-callout-tip-color-frame, coltitle=black, titlerule=0mm, leftrule=.75mm, toprule=.15mm, bottomtitle=1mm, opacityback=0, title=\textcolor{quarto-callout-tip-color}{\faLightbulb}\hspace{0.5em}{Truc}, toptitle=1mm, bottomrule=.15mm, opacitybacktitle=0.6, colback=white]

Les mots de 8 ou de 16 bits écrits en binaire sont plus lisibles si on
les inscrit en laissant un espace entre les groupes de quatre bits comme
ceci: 0100 0001

\end{tcolorbox}

\begin{tcolorbox}[enhanced jigsaw, arc=.35mm, breakable, rightrule=.15mm, left=2mm, colbacktitle=quarto-callout-tip-color!10!white, colframe=quarto-callout-tip-color-frame, coltitle=black, titlerule=0mm, leftrule=.75mm, toprule=.15mm, bottomtitle=1mm, opacityback=0, title=\textcolor{quarto-callout-tip-color}{\faLightbulb}\hspace{0.5em}{Truc}, toptitle=1mm, bottomrule=.15mm, opacitybacktitle=0.6, colback=white]

On a avantage à représenter les zéros non significatifs pour montrer la
taille des codes transcrits. remarquez que ces 0 à gauche ne sont
d'ailleurs pas toujours \emph{non significatifs}. En effet, les codes
binaires ne représentent pas toujours des valurs numériques. Ce sont
parfois simplement des codes qui ne représentent pas des quantités.
Inutile donc de faire de l'arithmétique avec ces codes. Dans ce cas,
cela n'a aucun sens de vouloir les convertir en décimal et ce serait une
erreur d'omettre l'écriture des zéros à gauche.

\end{tcolorbox}

\hypertarget{pruxe9fixes-pour-repruxe9senter-les-puissances-de-10-dans-la-vie-courante}{%
\subsubsection{Préfixes pour représenter les puissances de 10 dans la
vie
courante}\label{pruxe9fixes-pour-repruxe9senter-les-puissances-de-10-dans-la-vie-courante}}

On doit souvent en informatique utiliser de très grands nombres. Le
système métrique nous a habitués à utiliser des puissances de 10 et de
regrouper les chiffres par trois comme 1 000 = 10\textsuperscript{3} ou
1 000 000 = 10\textsuperscript{6}.

Pour les grands nombres, les puissances successives de
10\textsuperscript{3} portent ces noms:

\begin{longtable}[]{@{}cc@{}}
\toprule()
\textbf{Préfixe} & \textbf{Puissance} \\
\midrule()
\endhead
\textbf{Kilo (k)} & 10\textsuperscript{3} \\
\textbf{Méga (M)} & 10\textsuperscript{6} \\
\textbf{Giga (G)} & 10\textsuperscript{9} \\
\textbf{Téra (T)} & 10\textsuperscript{12} \\
\textbf{Péta (P)} & 10\textsuperscript{15} \\
\textbf{Exa (E)} & 10\textsuperscript{18} \\
\bottomrule()
\end{longtable}

Les petits nombres quant à eux s'expriment au moyen des puisances de
10\textsuperscript{-3}:

\begin{longtable}[]{@{}cc@{}}
\toprule()
\textbf{Préfixe} & \textbf{Puissance} \\
\midrule()
\endhead
\textbf{milli (m)} & 10\textsuperscript{-3} \\
\textbf{micro (\(\mu\))} & 10\textsuperscript{-6} \\
\textbf{nano (n)} & 10\textsuperscript{-9} \\
\textbf{pico (p)} & 10\textsuperscript{-12} \\
\textbf{femto (f)} & 10\textsuperscript{-15} \\
\bottomrule()
\end{longtable}

En informatique, on n'utilise pas les puissances de
10\textsuperscript{3} pour désigner des nombres mais plutôt les
puissances de 2\textsuperscript{10}. En effet, 2\textsuperscript{10} =
1024. Ce nombre étant proche de 1000, il est désigné par le préfixe
\emph{kilo}.

\begin{tcolorbox}[enhanced jigsaw, arc=.35mm, breakable, rightrule=.15mm, left=2mm, colbacktitle=quarto-callout-note-color!10!white, colframe=quarto-callout-note-color-frame, coltitle=black, titlerule=0mm, leftrule=.75mm, toprule=.15mm, bottomtitle=1mm, opacityback=0, title=\textcolor{quarto-callout-note-color}{\faInfo}\hspace{0.5em}{Note}, toptitle=1mm, bottomrule=.15mm, opacitybacktitle=0.6, colback=white]

Quand il s'agit de dimension de mémoires, on parle

\end{tcolorbox}

\hypertarget{les-nombres-hexaduxe9cimaux}{%
\subsection{Les nombres
hexadécimaux}\label{les-nombres-hexaduxe9cimaux}}

\hypertarget{addition-en-binaire}{%
\section{Addition en binaire}\label{addition-en-binaire}}

\hypertarget{repruxe9sentation-des-entiers-1}{%
\section{Représentation des
entiers}\label{repruxe9sentation-des-entiers-1}}

\hypertarget{arithmuxe9tique-de-lordinateur-1}{%
\section{Arithmétique de
l'ordinateur}\label{arithmuxe9tique-de-lordinateur-1}}

\hypertarget{nombres-en-virgules-flottantes}{%
\section{Nombres en virgules
flottantes}\label{nombres-en-virgules-flottantes}}

\hypertarget{lies-my-computer-tell-me}{%
\section{Lies my computer tell me}\label{lies-my-computer-tell-me}}

\bookmarksetup{startatroot}

\hypertarget{logique}{%
\chapter{Logique}\label{logique}}

\hypertarget{logique-propositionnelle}{%
\section{Logique propositionnelle}\label{logique-propositionnelle}}

\leavevmode\vadjust pre{\hypertarget{def-proposition}{}}%
\begin{definition}[Proposition]\label{def-proposition}

Un énoncé qui est soit vrai, soit faux est appelé une
\textbf{proposition}. La \textbf{valeur de vérité} d'une proposition est
donc \textbf{VRAI} ou \textbf{FAUX}.

En \texttt{Python}, les valeurs de vérités sont données par
\texttt{True} (\textbf{VRAI}) et \texttt{False} (\textbf{FAUX}).

\end{definition}

Un énoncé qui n'est pas une proposition (comme un paradoxe, une phrase
impérative ou interrogative) sera qualifié d'innaceptable.

\leavevmode\vadjust pre{\hypertarget{exm-propositions}{}}%
\begin{example}[]\label{exm-propositions}

Les énoncés suivants sont des propositions:

\begin{itemize}
\tightlist
\item
  Les numéros de téléphones au Canada ont dix chiffres.
\item
  La lune est faite de fromage.
\item
  42 est la réponse à la question portant sur la \emph{vie, l'univers et
  tout ce qui existe}.
\item
  Chaque nombre pair plus grand que 2 peut être exprimé comme la somme
  de deux nombres premiers.
\item
  \(3+7=12\)
\end{itemize}

Les énoncés suivants ne sont \textbf{pas} des propositions:

\begin{itemize}
\tightlist
\item
  Voulez-vous du gâteau?
\item
  La somme de deux carrés.
\item
  \(1+3+5+7+\ldots +2n+1\).
\item
  Va dans ta chambre!
\item
  \(3+x=12\)
\end{itemize}

\end{example}

Nous utilisons une table de vérité pour montrer les valeurs de vérité de
propositions composées.

\hypertarget{la-nuxe9gation}{%
\subsection{La négation}\label{la-nuxe9gation}}

\leavevmode\vadjust pre{\hypertarget{def-negation}{}}%
\begin{definition}[La négation]\label{def-negation}

Soit \(p\) une proposition. L'énoncé:

\begin{quote}
\emph{Il n'est pas vrai que \(p\).}
\end{quote}

est une autre proposition appelée \textbf{négation} de \(p\), qui est
représentée par \(\lnot p\). La proposition \(\lnot p\) se lit \emph{non
\(p\)}. La table de vérité de la négation est donnée ci-dessous.

\begin{longtable}[]{@{}cc@{}}
\toprule()
\(p\) & \(\lnot p\) \\
\midrule()
\endhead
\(\phantom{V}\) & \(\phantom{V}\) \\
\(\phantom{V}\) & \(\phantom{V}\) \\
\bottomrule()
\end{longtable}

En \texttt{Python}, l'opérateur \texttt{not} permet de faire la négation
d'une valeur de vérité.

\hypertarget{negation-python}{}
\begin{Shaded}
\begin{Highlighting}[]
\KeywordTok{def}\NormalTok{ negation(p):}
    \ControlFlowTok{return} \KeywordTok{not}\NormalTok{ p}

\BuiltInTok{print}\NormalTok{(}\StringTok{"p    non\_p"}\NormalTok{)}
\ControlFlowTok{for}\NormalTok{ p }\KeywordTok{in}\NormalTok{ [}\VariableTok{True}\NormalTok{, }\VariableTok{False}\NormalTok{]:}
\NormalTok{    non\_p }\OperatorTok{=}\NormalTok{ negation(p)}
    \BuiltInTok{print}\NormalTok{(p, non\_p)}
\end{Highlighting}
\end{Shaded}

\begin{verbatim}
p    non_p
True False
False True
\end{verbatim}

\end{definition}

\hypertarget{la-conjonction}{%
\subsection{La conjonction}\label{la-conjonction}}

\begin{quote}
Je suis une roche \textbf{ET} je suis une île.
\end{quote}

\leavevmode\vadjust pre{\hypertarget{def-conjonction}{}}%
\begin{definition}[La conjonction]\label{def-conjonction}

Soit \(p\) et \(q\) deux propositions. La proposition \emph{\(p\) et
\(q\)}, notée \(p\wedge q\), est vraie si à la fois \(p\) et \(q\) sont
vraies. Elle est fausse dans tous les autres cas. Cette proposition est
appelée la \textbf{conjonction} de \(p\) et de \(q\). La table de vérité
de la conjonction est donnée ci-dessous.

\begin{longtable}[]{@{}ccc@{}}
\toprule()
\(p\) & \(q\) & \(p \wedge q\) \\
\midrule()
\endhead
\(\phantom{V}\) & \(\phantom{V}\) & \(\phantom{V}\) \\
\(\phantom{V}\) & \(\phantom{V}\) & \(\phantom{V}\) \\
\(\phantom{V}\) & \(\phantom{V}\) & \(\phantom{V}\) \\
\(\phantom{V}\) & \(\phantom{V}\) & \(\phantom{V}\) \\
\bottomrule()
\end{longtable}

En \texttt{Python}, l'opérateur \texttt{and} permet de faire la
conjonction de deux valeurs de vérité.

\hypertarget{conjonction-python}{}
\begin{Shaded}
\begin{Highlighting}[]
\KeywordTok{def}\NormalTok{ conjonction(p, q):}
    \ControlFlowTok{return}\NormalTok{ p }\KeywordTok{and}\NormalTok{ q}

\BuiltInTok{print}\NormalTok{(}\StringTok{"p    q    p\_et\_q"}\NormalTok{)}
\ControlFlowTok{for}\NormalTok{ p }\KeywordTok{in}\NormalTok{ [}\VariableTok{True}\NormalTok{, }\VariableTok{False}\NormalTok{]:}
    \ControlFlowTok{for}\NormalTok{ q }\KeywordTok{in}\NormalTok{ [}\VariableTok{True}\NormalTok{, }\VariableTok{False}\NormalTok{]:}
\NormalTok{        p\_et\_q }\OperatorTok{=}\NormalTok{ conjonction(p, q)}
        \BuiltInTok{print}\NormalTok{(p, q, p\_et\_q)}
\end{Highlighting}
\end{Shaded}

\begin{verbatim}
p    q    p_et_q
True True True
True False False
False True False
False False False
\end{verbatim}

\end{definition}

\hypertarget{la-disjonction}{%
\subsection{La disjonction}\label{la-disjonction}}

\begin{quote}
Elle a étudié très fort \textbf{OU} elle est extrêmement brillante.
\end{quote}

\leavevmode\vadjust pre{\hypertarget{def-disjonction}{}}%
\begin{definition}[La disjonction]\label{def-disjonction}

Soit \(p\) et \(q\) deux propositions. La proposition \emph{\(p\) ou
\(q\)}, notée \(p\vee q\), est fausse si \(p\) et \(q\) sont fausses.
Elle est vraie dans tous les autres cas. Cette proposition est appelée
la \textbf{disjonction} de \(p\) et de \(q\). La table de vérité de la
disjonction est donnée ci-dessous.

\begin{longtable}[]{@{}ccc@{}}
\toprule()
\(p\) & \(q\) & \(p \vee q\) \\
\midrule()
\endhead
\(\phantom{V}\) & \(\phantom{V}\) & \(\phantom{V}\) \\
\(\phantom{V}\) & \(\phantom{V}\) & \(\phantom{V}\) \\
\(\phantom{V}\) & \(\phantom{V}\) & \(\phantom{V}\) \\
\(\phantom{V}\) & \(\phantom{V}\) & \(\phantom{V}\) \\
\bottomrule()
\end{longtable}

En \texttt{Python}, l'opérateur \texttt{or} permet de faire la
disjonction de deux valeurs de vérité.

\hypertarget{disjonction-python}{}
\begin{Shaded}
\begin{Highlighting}[]
\KeywordTok{def}\NormalTok{ disjonction(p, q):}
    \ControlFlowTok{return}\NormalTok{ p }\KeywordTok{or}\NormalTok{ q}

\BuiltInTok{print}\NormalTok{(}\StringTok{"p    q    p\_ou\_q"}\NormalTok{)}
\ControlFlowTok{for}\NormalTok{ p }\KeywordTok{in}\NormalTok{ [}\VariableTok{True}\NormalTok{, }\VariableTok{False}\NormalTok{]:}
    \ControlFlowTok{for}\NormalTok{ q }\KeywordTok{in}\NormalTok{ [}\VariableTok{True}\NormalTok{, }\VariableTok{False}\NormalTok{]:}
\NormalTok{        p\_ou\_q }\OperatorTok{=}\NormalTok{ disjonction(p, q)}
        \BuiltInTok{print}\NormalTok{(p, q, p\_ou\_q)}
\end{Highlighting}
\end{Shaded}

\begin{verbatim}
p    q    p_ou_q
True True True
True False True
False True True
False False False
\end{verbatim}

\end{definition}

\hypertarget{la-disjonction-exclusive}{%
\subsection{La disjonction exclusive}\label{la-disjonction-exclusive}}

\begin{quote}
Prenez \textbf{SOIT} deux Advil \textbf{OU} deux Tylenols.
\end{quote}

\leavevmode\vadjust pre{\hypertarget{def-disjonction-exclusive}{}}%
\begin{definition}[La disjonction
exclusive]\label{def-disjonction-exclusive}

Soit \(p\) et \(q\) deux propositions. La proposition \emph{\(p\) ou
exclusif \(q\)}, notée \(p\oplus q\), est vraie si \(p\) et \(q\) ont
des valeurs de vérité \textbf{différentes}. Elle est fausse dans tous
les autres cas. Cette proposition est appelée la \textbf{disjonction
exclusive} de \(p\) et de \(q\). La table de vérité de la disjonction
exclusive est donnée ci-dessous.

\begin{longtable}[]{@{}ccc@{}}
\toprule()
\(p\) & \(q\) & \(p \oplus q\) \\
\midrule()
\endhead
\(\phantom{V}\) & \(\phantom{V}\) & \(\phantom{V}\) \\
\(\phantom{V}\) & \(\phantom{V}\) & \(\phantom{V}\) \\
\(\phantom{V}\) & \(\phantom{V}\) & \(\phantom{V}\) \\
\(\phantom{V}\) & \(\phantom{V}\) & \(\phantom{V}\) \\
\bottomrule()
\end{longtable}

En \texttt{Python}, il n'existe pas d'opérateur logique pour effectuer
la disjonction exclusive. On peut par contre utiliser l'opérateur bit à
bit \texttt{\^{}} pour faire cette disjonction exclusive.

\end{definition}

\leavevmode\vadjust pre{\hypertarget{exm-disjonction-exclusive-python}{}}%
\begin{example}[]\label{exm-disjonction-exclusive-python}

Utilisez les opérateurs logiques vus précédemment pour construire la
table de vérité de la disjonction exclusive dans \texttt{Python}.

\hypertarget{disjonction-exclusive-python-todo}{}
\begin{Shaded}
\begin{Highlighting}[]
\KeywordTok{def}\NormalTok{ disjonction\_exclusive(p, q):}
    \ControlFlowTok{return} \CommentTok{\#REMPLACEZ MOI\#}

\BuiltInTok{print}\NormalTok{(}\StringTok{"p    q    p\_ou\_exclusif\_q"}\NormalTok{)}
\ControlFlowTok{for}\NormalTok{ p }\KeywordTok{in}\NormalTok{ [}\VariableTok{True}\NormalTok{, }\VariableTok{False}\NormalTok{]:}
    \ControlFlowTok{for}\NormalTok{ q }\KeywordTok{in}\NormalTok{ [}\VariableTok{True}\NormalTok{, }\VariableTok{False}\NormalTok{]:}
\NormalTok{        p\_ou\_exclusif\_q }\OperatorTok{=}\NormalTok{ disjonction\_exclusive(p, q)}
        \BuiltInTok{print}\NormalTok{(p, q, p\_ou\_exclusif\_q)}
\end{Highlighting}
\end{Shaded}

\hypertarget{disjonction-exclusive-python}{}
\begin{verbatim}
p    q    p_ou_exclusif_q
True True False
True False True
False True True
False False False
\end{verbatim}

\end{example}

\begin{tcolorbox}[enhanced jigsaw, arc=.35mm, breakable, rightrule=.15mm, left=2mm, colbacktitle=quarto-callout-important-color!10!white, colframe=quarto-callout-important-color-frame, coltitle=black, titlerule=0mm, leftrule=.75mm, toprule=.15mm, bottomtitle=1mm, opacityback=0, title=\textcolor{quarto-callout-important-color}{\faExclamation}\hspace{0.5em}{Important}, toptitle=1mm, bottomrule=.15mm, opacitybacktitle=0.6, colback=white]

La disjonction exclusive signifie l'un ou l'autre, mais pas les deux.

\end{tcolorbox}

\hypertarget{limplication}{%
\subsection{L'implication}\label{limplication}}

\begin{quote}
\textbf{SI} vous avez 100 à l'examen final, \textbf{ALORS} vous
obtiendrez A dans ce cours.
\end{quote}

\leavevmode\vadjust pre{\hypertarget{def-implication}{}}%
\begin{definition}[L'implication]\label{def-implication}

Soit \(p\) et \(q\) deux propositions. L'\textbf{implication}
\(p\rightarrow q\) est une proposition qui est fausse quand \(p\) est
vraie et que \(q\) est fausse, et qui est vraie dans tous les autres
cas. Dans une implication, \(p\) est appelée l'\textbf{hypothèse} (ou
l'\textbf{antécédent} ou la \textbf{prémisse}) et \(q\), la
\textbf{conclusion} (ou la \textbf{conséquence}). La table de vérité de
l'implication' est donnée ci-dessous.

\begin{longtable}[]{@{}ccc@{}}
\toprule()
\(p\) & \(q\) & \(p \rightarrow q\) \\
\midrule()
\endhead
\(\phantom{V}\) & \(\phantom{V}\) & \(\phantom{V}\) \\
\(\phantom{V}\) & \(\phantom{V}\) & \(\phantom{V}\) \\
\(\phantom{V}\) & \(\phantom{V}\) & \(\phantom{V}\) \\
\(\phantom{V}\) & \(\phantom{V}\) & \(\phantom{V}\) \\
\bottomrule()
\end{longtable}

En \texttt{Python}, il n'existe pas d'opérateur logique pour effectuer
l'implication.

\end{definition}

\leavevmode\vadjust pre{\hypertarget{exm-implication-python}{}}%
\begin{example}[]\label{exm-implication-python}

Utilisez les opérateurs logiques vus précédemment pour construire la
table de vérité de l'implication dans \texttt{Python}.

\hypertarget{implication-python-todo}{}
\begin{Shaded}
\begin{Highlighting}[]
\KeywordTok{def}\NormalTok{ implication(p, q):}
    \ControlFlowTok{return} \CommentTok{\#REMPLACEZ MOI\#}

\BuiltInTok{print}\NormalTok{(}\StringTok{"p    q    p\_implique\_q"}\NormalTok{)}
\ControlFlowTok{for}\NormalTok{ p }\KeywordTok{in}\NormalTok{ [}\VariableTok{True}\NormalTok{, }\VariableTok{False}\NormalTok{]:}
    \ControlFlowTok{for}\NormalTok{ q }\KeywordTok{in}\NormalTok{ [}\VariableTok{True}\NormalTok{, }\VariableTok{False}\NormalTok{]:}
\NormalTok{        p\_implique\_q }\OperatorTok{=}\NormalTok{ implication(p, q)}
        \BuiltInTok{print}\NormalTok{(p, q, p\_implique\_q)}
\end{Highlighting}
\end{Shaded}

\hypertarget{implication-python}{}
\begin{verbatim}
p    q    p_implique_q
True True True
True False False
False True True
False False True
\end{verbatim}

\end{example}

\begin{tcolorbox}[enhanced jigsaw, arc=.35mm, breakable, rightrule=.15mm, left=2mm, colbacktitle=quarto-callout-important-color!10!white, colframe=quarto-callout-important-color-frame, coltitle=black, titlerule=0mm, leftrule=.75mm, toprule=.15mm, bottomtitle=1mm, opacityback=0, title=\textcolor{quarto-callout-important-color}{\faExclamation}\hspace{0.5em}{Important}, toptitle=1mm, bottomrule=.15mm, opacitybacktitle=0.6, colback=white]

Une implication peut être considérée comme un \textbf{contrat} qui
échoue seulement si les conditions du contrat sont respectées mais les
résultats ne sont pas remplis.

\end{tcolorbox}

Comme les implications apparaissent constamment en mathématiques, il
existe une vaste terminologie pour désigner \(p\rightarrow q\). Voici
les modes les plus courants:

\begin{itemize}
\tightlist
\item
  si \(p\) alors \(q\);
\item
  \(p\) implique \(q\);
\item
  \(p\) seulement si \(q\);
\item
  \(p\) est suffisant pour \(q\);
\item
  \(q\) si \(p\);
\item
  \(q\) chaque fois que \(p\);
\item
  \(q\) est nécessaire à \(p\).
\end{itemize}

\hypertarget{la-biconditionnelle}{%
\subsection{La biconditionnelle}\label{la-biconditionnelle}}

\begin{quote}
Il pleut dehors \textbf{SI ET SEULEMENT SI} c'est un jour nuageux.
\end{quote}

\leavevmode\vadjust pre{\hypertarget{def-biconditionnelle}{}}%
\begin{definition}[La biconditionnelle]\label{def-biconditionnelle}

Soit \(p\) et \(q\) deux propositions. La \textbf{biconditionnelle}
\(p\leftrightarrow q\) est une proposition qui est vraie quand \(p\) et
\(q\) ont les mêmes valeurs de vérité et qui est fausse dans les autres
cas. La table de vérité de la biconditionnelle est donnée ci-dessous.

\begin{longtable}[]{@{}ccc@{}}
\toprule()
\(p\) & \(q\) & \(p \leftrightarrow q\) \\
\midrule()
\endhead
\(\phantom{V}\) & \(\phantom{V}\) & \(\phantom{V}\) \\
\(\phantom{V}\) & \(\phantom{V}\) & \(\phantom{V}\) \\
\(\phantom{V}\) & \(\phantom{V}\) & \(\phantom{V}\) \\
\(\phantom{V}\) & \(\phantom{V}\) & \(\phantom{V}\) \\
\bottomrule()
\end{longtable}

En \texttt{Python}, il n'existe pas d'opérateur logique pour effectuer
la biconditionnelle.

\end{definition}

\leavevmode\vadjust pre{\hypertarget{exm-biconditionnelle-python}{}}%
\begin{example}[]\label{exm-biconditionnelle-python}

Utilisez les opérateurs logiques vus précédemment pour construire la
table de vérité de la biconditionnelle dans \texttt{Python}.

\hypertarget{biconditionnelle-python-todo}{}
\begin{Shaded}
\begin{Highlighting}[]
\KeywordTok{def}\NormalTok{ biconditionnelle(p, q):}
    \ControlFlowTok{return} \CommentTok{\#REMPLACEZ MOI\#}

\BuiltInTok{print}\NormalTok{(}\StringTok{"p    q    p\_biconditionnelle\_q"}\NormalTok{)}
\ControlFlowTok{for}\NormalTok{ p }\KeywordTok{in}\NormalTok{ [}\VariableTok{True}\NormalTok{, }\VariableTok{False}\NormalTok{]:}
    \ControlFlowTok{for}\NormalTok{ q }\KeywordTok{in}\NormalTok{ [}\VariableTok{True}\NormalTok{, }\VariableTok{False}\NormalTok{]:}
\NormalTok{        p\_biconditionnelle\_q }\OperatorTok{=}\NormalTok{ biconditionnelle(p, q)}
        \BuiltInTok{print}\NormalTok{(p, q, p\_biconditionnelle\_q)}
\end{Highlighting}
\end{Shaded}

\hypertarget{biconditionnelle-python}{}
\begin{verbatim}
p    q    p_biconditionnelle_q
True True True
True False False
False True False
False False True
\end{verbatim}

\end{example}

\begin{tcolorbox}[enhanced jigsaw, arc=.35mm, breakable, rightrule=.15mm, left=2mm, colbacktitle=quarto-callout-important-color!10!white, colframe=quarto-callout-important-color-frame, coltitle=black, titlerule=0mm, leftrule=.75mm, toprule=.15mm, bottomtitle=1mm, opacityback=0, title=\textcolor{quarto-callout-important-color}{\faExclamation}\hspace{0.5em}{Important}, toptitle=1mm, bottomrule=.15mm, opacitybacktitle=0.6, colback=white]

La biconditionnelle est vraie si les propositions ont la même valeur de
vérité et fausse autrement.

\end{tcolorbox}

Comme les biconditionnelles apparaissent constamment en mathématiques,
il existe une vaste terminologie pour désigner \(p\leftrightarrow q\).
Voici les modes les plus courants:

\begin{itemize}
\tightlist
\item
  \(p\) si et seulement si \(q\);
\item
  \(p\) est nécessaire et suffisante pour \(q\);
\item
  si \(p\) alors \(q\) et réciproquement.
\end{itemize}

\leavevmode\vadjust pre{\hypertarget{def-reciproque-contraposee-inverse}{}}%
\begin{definition}[Réciproque, contraposée et
inverse]\label{def-reciproque-contraposee-inverse}

~

\begin{itemize}
\tightlist
\item
  La \textbf{réciproque} de la proposition \(p\rightarrow q\) est la
  proposition \(q \rightarrow p\).
\item
  La \textbf{contraposée} de la proposition \(p\rightarrow q\) est la
  proposition \(\lnot q \rightarrow \lnot p\).
\item
  L'\textbf{inverse} de la proposition \(p\rightarrow q\) est la
  proposition \(\lnot p \rightarrow \lnot q\).
\end{itemize}

\end{definition}

\hypertarget{uxe9quivalences-propositionnelles}{%
\section{Équivalences
propositionnelles}\label{uxe9quivalences-propositionnelles}}

Une proposition composée est une proposition formée de plusieurs
connecteurs logiques.

\leavevmode\vadjust pre{\hypertarget{def-tautologie-contradiction-contingence}{}}%
\begin{definition}[Tautologie, contradiction et
contingence]\label{def-tautologie-contradiction-contingence}

Une proposition composée qui est toujours vraie, quelle que soit la
valeur de vérité des fonctions qui la compose est appelée une
\textbf{tautologie}. Une proposition composée qui est toujours fausse
est appelée une \textbf{contradiction}. Finalement, une proposition qui
n'est ni une tautologie ni une contradiction est appelée une
\textbf{contingence}.

\end{definition}

\leavevmode\vadjust pre{\hypertarget{exm-tautologie-contradiction}{}}%
\begin{example}[]\label{exm-tautologie-contradiction}

Remplissez la table de vérité suivante et dites si les propositions
composées sont des tautologies, des contradictions ou des contingences.

\begin{longtable}[]{@{}
  >{\centering\arraybackslash}p{(\columnwidth - 6\tabcolsep) * \real{0.2206}}
  >{\centering\arraybackslash}p{(\columnwidth - 6\tabcolsep) * \real{0.2206}}
  >{\centering\arraybackslash}p{(\columnwidth - 6\tabcolsep) * \real{0.2647}}
  >{\centering\arraybackslash}p{(\columnwidth - 6\tabcolsep) * \real{0.2941}}@{}}
\toprule()
\begin{minipage}[b]{\linewidth}\centering
\(p\)
\end{minipage} & \begin{minipage}[b]{\linewidth}\centering
\(q\)
\end{minipage} & \begin{minipage}[b]{\linewidth}\centering
\(p \vee \lnot p\)
\end{minipage} & \begin{minipage}[b]{\linewidth}\centering
\(p \wedge \lnot p\)
\end{minipage} \\
\midrule()
\endhead
\(\phantom{V}\) & \(\phantom{V}\) & \(\phantom{V}\) & \(\phantom{V}\) \\
\(\phantom{V}\) & \(\phantom{V}\) & \(\phantom{V}\) & \(\phantom{V}\) \\
\bottomrule()
\end{longtable}

\end{example}

\leavevmode\vadjust pre{\hypertarget{exm-proposition-compose}{}}%
\begin{example}[]\label{exm-proposition-compose}

Le code ci-dessous révèle la table de vérité de la proposition composée
\((p \wedge q) \vee \lnot q\).

\hypertarget{prop-composee-1}{}
\begin{Shaded}
\begin{Highlighting}[]
\KeywordTok{def}\NormalTok{ conjonction(p, q):}
    \ControlFlowTok{return}\NormalTok{ p }\KeywordTok{and}\NormalTok{ q}

\KeywordTok{def}\NormalTok{ disjonction(p, q):}
    \ControlFlowTok{return}\NormalTok{ p }\KeywordTok{or}\NormalTok{ q}

\BuiltInTok{print}\NormalTok{(}\StringTok{"p    q    a"}\NormalTok{)}
\ControlFlowTok{for}\NormalTok{ p }\KeywordTok{in}\NormalTok{ [}\VariableTok{True}\NormalTok{, }\VariableTok{False}\NormalTok{]:}
    \ControlFlowTok{for}\NormalTok{ q }\KeywordTok{in}\NormalTok{ [}\VariableTok{True}\NormalTok{, }\VariableTok{False}\NormalTok{]:}
\NormalTok{        a }\OperatorTok{=}\NormalTok{ disjonction(conjonction(p, q), }\KeywordTok{not}\NormalTok{ q)}
        \BuiltInTok{print}\NormalTok{(p, q, a)}
\end{Highlighting}
\end{Shaded}

\begin{verbatim}
p    q    a
True True True
True False True
False True False
False False True
\end{verbatim}

De quelle manière pouvez-vous modifier le code précédent pour obtenir la
table de vérité de la proposition composée
\((p \vee \lnot q) \wedge \lnot p\)?

\end{example}

Lorsque vous créez votre table de vérité, il est crucial que vous soyiez
systématique pour vous assurer d'avoir toutes les valeurs de vérité
possibles pour chacune des propositions simples. Chaque proposition a
deux valeurs de vérité possibles, le nombre de lignes de la table
devrait être égal à \(2^n\), où \(n\) est le nombre de propositions.
Vous devriez également considérer de briser vos propositions complexes
en plus petites propositions.

\leavevmode\vadjust pre{\hypertarget{exm-bloc-code}{}}%
\begin{example}[]\label{exm-bloc-code}

L'extrait de code suivant fait intervenir les variables booléennes
\(p\), \(q\) et \(r\). Chacune de ces variables peut prendre les valeurs
\textbf{vrai} ou \textbf{faux}. Pour chaque bloc indiqué, donnez toutes
les valeurs possibles pour \(p\), \(q\) et \(r\) au moment où le bloc
est atteint.

\hypertarget{bloc-code-python}{}
\begin{Shaded}
\begin{Highlighting}[]
\ControlFlowTok{if}\NormalTok{ (p }\KeywordTok{and}\NormalTok{ q):}
    \ControlFlowTok{if}\NormalTok{ r:}
        \CommentTok{\#BLOC 1\#}
    \ControlFlowTok{else}\NormalTok{:}
        \CommentTok{\#BLOC 2\#}
\ControlFlowTok{else}\NormalTok{:}
    \CommentTok{\#BLOC 3\#}
\end{Highlighting}
\end{Shaded}

\begin{longtable}[]{@{}
  >{\centering\arraybackslash}p{(\columnwidth - 10\tabcolsep) * \real{0.1061}}
  >{\centering\arraybackslash}p{(\columnwidth - 10\tabcolsep) * \real{0.1061}}
  >{\centering\arraybackslash}p{(\columnwidth - 10\tabcolsep) * \real{0.1061}}
  >{\centering\arraybackslash}p{(\columnwidth - 10\tabcolsep) * \real{0.2273}}
  >{\centering\arraybackslash}p{(\columnwidth - 10\tabcolsep) * \real{0.2273}}
  >{\centering\arraybackslash}p{(\columnwidth - 10\tabcolsep) * \real{0.2273}}@{}}
\toprule()
\begin{minipage}[b]{\linewidth}\centering
\(p\)
\end{minipage} & \begin{minipage}[b]{\linewidth}\centering
\(q\)
\end{minipage} & \begin{minipage}[b]{\linewidth}\centering
\(r\)
\end{minipage} & \begin{minipage}[b]{\linewidth}\centering
\(\phantom{V}\)
\end{minipage} & \begin{minipage}[b]{\linewidth}\centering
\(\phantom{V}\)
\end{minipage} & \begin{minipage}[b]{\linewidth}\centering
\(\phantom{V}\)
\end{minipage} \\
\midrule()
\endhead
\textbf{V} & \textbf{V} & \textbf{V} & & & \\
\textbf{V} & \textbf{V} & \textbf{F} & & & \\
\textbf{V} & \textbf{F} & \textbf{V} & & & \\
\textbf{V} & \textbf{F} & \textbf{F} & & & \\
\textbf{F} & \textbf{V} & \textbf{V} & & & \\
\textbf{F} & \textbf{V} & \textbf{F} & & & \\
\textbf{F} & \textbf{F} & \textbf{V} & & & \\
\textbf{F} & \textbf{F} & \textbf{F} & & & \\
\bottomrule()
\end{longtable}

\end{example}

\leavevmode\vadjust pre{\hypertarget{def-propositions-equivalentes}{}}%
\begin{definition}[Équivalences de
propositions]\label{def-propositions-equivalentes}

Les propositions \(p\) et \(q\) sont dites \textbf{logiquement
équivalentes} si la proposition \(p \leftrightarrow q\) est une
tautologie. Ainsi, deux propositions sont logiquement équivalentes si
elles ont la même table de vérité, c'est-à-dire la même valeur de vérité
dans tous les cas possibles.

La notation \(p\equiv q\) signifie que \(p\) et \(q\) sont équivalentes.

\end{definition}

\leavevmode\vadjust pre{\hypertarget{exm-proposition-equivalentes-1}{}}%
\begin{example}[]\label{exm-proposition-equivalentes-1}

Vérifiez l'équivalence suivante à l'aide d'une table de vérité. \[
p \rightarrow q \equiv \lnot p \vee q
\]

\begin{longtable}[]{@{}ccccc@{}}
\toprule()
\(p\) & \(q\) & \(\phantom{V}\) & \(\phantom{V}\) & \(\phantom{V}\) \\
\midrule()
\endhead
\textbf{V} & \textbf{V} & & & \\
\textbf{V} & \textbf{F} & & & \\
\textbf{F} & \textbf{V} & & & \\
\textbf{F} & \textbf{F} & & & \\
\bottomrule()
\end{longtable}

\end{example}

\leavevmode\vadjust pre{\hypertarget{exm-proposition-equivalentes-2}{}}%
\begin{example}[]\label{exm-proposition-equivalentes-2}

Vérifiez l'équivalence suivante à l'aide d'une table de vérité. \[
\lnot (p \vee q) \equiv \lnot p \wedge \lnot q
\]

\begin{longtable}[]{@{}
  >{\centering\arraybackslash}p{(\columnwidth - 12\tabcolsep) * \real{0.0787}}
  >{\centering\arraybackslash}p{(\columnwidth - 12\tabcolsep) * \real{0.0787}}
  >{\centering\arraybackslash}p{(\columnwidth - 12\tabcolsep) * \real{0.1685}}
  >{\centering\arraybackslash}p{(\columnwidth - 12\tabcolsep) * \real{0.1685}}
  >{\centering\arraybackslash}p{(\columnwidth - 12\tabcolsep) * \real{0.1685}}
  >{\centering\arraybackslash}p{(\columnwidth - 12\tabcolsep) * \real{0.1685}}
  >{\centering\arraybackslash}p{(\columnwidth - 12\tabcolsep) * \real{0.1685}}@{}}
\toprule()
\begin{minipage}[b]{\linewidth}\centering
\(p\)
\end{minipage} & \begin{minipage}[b]{\linewidth}\centering
\(q\)
\end{minipage} & \begin{minipage}[b]{\linewidth}\centering
\(\phantom{V}\)
\end{minipage} & \begin{minipage}[b]{\linewidth}\centering
\(\phantom{V}\)
\end{minipage} & \begin{minipage}[b]{\linewidth}\centering
\(\phantom{V}\)
\end{minipage} & \begin{minipage}[b]{\linewidth}\centering
\(\phantom{V}\)
\end{minipage} & \begin{minipage}[b]{\linewidth}\centering
\(\phantom{V}\)
\end{minipage} \\
\midrule()
\endhead
\textbf{V} & \textbf{V} & & & & & \\
\textbf{V} & \textbf{F} & & & & & \\
\textbf{F} & \textbf{V} & & & & & \\
\textbf{F} & \textbf{F} & & & & & \\
\bottomrule()
\end{longtable}

\end{example}

Pour gagner du temps

\hypertarget{pruxe9dicats-et-quantificateurs}{%
\section{Prédicats et
quantificateurs}\label{pruxe9dicats-et-quantificateurs}}

\leavevmode\vadjust pre{\hypertarget{def-quantificateurs}{}}%
\begin{definition}[Quantificateurs]\label{def-quantificateurs}

\[
\forall:\ \text{quantificateur universel} \qquad \exists:\ \text{quantificateur existentiel}
\]

\end{definition}

\bookmarksetup{startatroot}

\hypertarget{thuxe9orie-des-ensembles}{%
\chapter{Théorie des ensembles}\label{thuxe9orie-des-ensembles}}

\hypertarget{notions-de-base-sur-les-ensembles}{%
\section{Notions de base sur les
ensembles}\label{notions-de-base-sur-les-ensembles}}

\hypertarget{ensembles-de-nombres-mathbbn-mathbbz-mathbbq-mathbbr}{%
\section{\texorpdfstring{Ensembles de nombres \(\mathbb{N}\),
\(\mathbb{Z}\), \(\mathbb{Q}\),
\(\mathbb{R}\)}{Ensembles de nombres \textbackslash mathbb\{N\}, \textbackslash mathbb\{Z\}, \textbackslash mathbb\{Q\}, \textbackslash mathbb\{R\}}}\label{ensembles-de-nombres-mathbbn-mathbbz-mathbbq-mathbbr}}

\hypertarget{produit-cartuxe9sien}{%
\section{Produit cartésien}\label{produit-cartuxe9sien}}

\hypertarget{opuxe9rations-sur-les-ensembles-cap-cup-oplus--}{%
\section{\texorpdfstring{Opérations sur les ensembles \(\cap\),
\(\cup\), \(\oplus\),
\(-\)}{Opérations sur les ensembles \textbackslash cap, \textbackslash cup, \textbackslash oplus, -}}\label{opuxe9rations-sur-les-ensembles-cap-cup-oplus--}}

\hypertarget{diagrammes-de-venn}{%
\section{Diagrammes de Venn}\label{diagrammes-de-venn}}

\hypertarget{repruxe9sentation-de-sous-ensembles-par-trains-de-bits}{%
\section{Représentation de sous-ensembles par trains de
bits}\label{repruxe9sentation-de-sous-ensembles-par-trains-de-bits}}

\hypertarget{polygones-convexes-avec-des-opuxe9rations-sur-les-ensembles}{%
\section{Polygones convexes avec des opérations sur les
ensembles}\label{polygones-convexes-avec-des-opuxe9rations-sur-les-ensembles}}

\bookmarksetup{startatroot}

\hypertarget{fonctions}{%
\chapter{Fonctions}\label{fonctions}}

\hypertarget{fonctions-plancher-et-plafond}{%
\section{Fonctions plancher et
plafond}\label{fonctions-plancher-et-plafond}}

UTILE LORSQUE NOUS FERONS DE LA THÉORIE DES NOMBRES

\hypertarget{fonctions-en-python}{%
\section{\texorpdfstring{Fonctions en
\texttt{Python}}{Fonctions en Python}}\label{fonctions-en-python}}

DEVRAIT-ON PARLER DE ÇA????

DICTIONNAIRE, HACHAGE\ldots{}

\leavevmode\vadjust pre{\hypertarget{exm-hash-python}{}}%
\begin{example}[]\label{exm-hash-python}

\href{https://www.wikiwand.com/en/Fowler\%E2\%80\%93Noll\%E2\%80\%93Vo_hash_function}{Fonction
de hachage dans Python}

\href{https://andrewbrookins.com/technology/pythons-default-hash-algorithm/}{Hachage
Python}

\href{https://thepythoncorner.com/posts/2020-08-21-hash-tables-understanding-dictionaries/}{Dictionnary
in Python}

A checksum is used to determine if something is the same.

If you have download a file, you can never be sure if it got corrupted
on the way to your machine. You can use cksum to calculate a checksum
(based on CRC-32) of the copy you now have and can then compare it to
the checksum the file should have. This is how you check for file
integrity.

A hash function is used to map data to other data of fixed size. A
perfect hash function is injective, so there are no collisions. Every
input has one fixed output.

A cryptographic hash function is used for verification. With a
cryptographic hash function you should to not be able to compute the
original input.

A very common use case is password hashing. This allows the verification
of a password without having to save the password itself. A service
provider only saves a hash of a password and is not able to compute the
original password. If the database of password hashes gets compromised,
an attacker should not be able to compute these passwords as well. This
is not the case, because there are strong and weak algorithms for
password hashing. You can find more on that on this very site.

TL;DR:

Checksums are used to compare two pieces of information to check if two
parties have exactly the same thing.

Hashes are used (in cryptography) to verify something, but this time,
deliberately only one party has access to the data that has to be
verified, while the other party only has access to the hash.

\end{example}

\hypertarget{injection-surjection-et-bijection}{%
\section{Injection, surjection et
bijection}\label{injection-surjection-et-bijection}}

\hypertarget{fonction-de-hachage-est-une-fonction-injective-surjective}{%
\subsection{Fonction de hachage est une fonction injective?
surjective?}\label{fonction-de-hachage-est-une-fonction-injective-surjective}}

\bookmarksetup{startatroot}

\hypertarget{notation-grand-o}{%
\chapter{Notation grand O}\label{notation-grand-o}}

\hypertarget{mesurer-un-temps-de-calcul-avec-une-fonction}{%
\section{Mesurer un temps de calcul avec une
fonction}\label{mesurer-un-temps-de-calcul-avec-une-fonction}}

\hypertarget{notation-grand-o-1}{%
\section{Notation grand-O}\label{notation-grand-o-1}}

\hypertarget{sommations}{%
\section{Sommations}\label{sommations}}

\hypertarget{uxe9tablir-la-complexituxe9-dun-algorithme}{%
\section{Établir la complexité d'un
algorithme}\label{uxe9tablir-la-complexituxe9-dun-algorithme}}

\hypertarget{calculabilituxe9-et-complexituxe9}{%
\section{Calculabilité et
complexité}\label{calculabilituxe9-et-complexituxe9}}

\hypertarget{p-vs-np}{%
\section{P vs NP}\label{p-vs-np}}

\bookmarksetup{startatroot}

\hypertarget{introduction-aux-algorithmes}{%
\chapter{Introduction aux
algorithmes}\label{introduction-aux-algorithmes}}

\hypertarget{bogo-sort}{%
\section{Bogo sort}\label{bogo-sort}}

\begin{Shaded}
\begin{Highlighting}[]
\ImportTok{from}\NormalTok{ random }\ImportTok{import}\NormalTok{ shuffle}
\ImportTok{from}\NormalTok{ random }\ImportTok{import}\NormalTok{ seed}
\ImportTok{from}\NormalTok{ random }\ImportTok{import}\NormalTok{ randint}

\KeywordTok{def}\NormalTok{ is\_sorted(data) }\OperatorTok{{-}\textgreater{}} \BuiltInTok{bool}\NormalTok{:}
    \CommentTok{"""Determine whether the data is sorted."""}
    \ControlFlowTok{return} \BuiltInTok{all}\NormalTok{(a }\OperatorTok{\textless{}=}\NormalTok{ b }\ControlFlowTok{for}\NormalTok{ a, b }\KeywordTok{in} \BuiltInTok{zip}\NormalTok{(data, data[}\DecValTok{1}\NormalTok{:]))}

\KeywordTok{def}\NormalTok{ bogosort(data) }\OperatorTok{{-}\textgreater{}} \BuiltInTok{list}\NormalTok{:}
    \CommentTok{"""Shuffle data until sorted."""}
\NormalTok{    N }\OperatorTok{=} \DecValTok{0}
    \ControlFlowTok{while} \KeywordTok{not}\NormalTok{ is\_sorted(data):}
\NormalTok{        shuffle(data)}
\NormalTok{        N }\OperatorTok{=}\NormalTok{ N }\OperatorTok{+} \DecValTok{1}
    \ControlFlowTok{return}\NormalTok{ data, N}

\NormalTok{seed(}\DecValTok{1234}\NormalTok{)}
\NormalTok{N }\OperatorTok{=} \DecValTok{8}
\NormalTok{data }\OperatorTok{=}\NormalTok{ [randint(}\DecValTok{1}\NormalTok{,}\DecValTok{10}\NormalTok{) }\ControlFlowTok{for}\NormalTok{ x }\KeywordTok{in} \BuiltInTok{range}\NormalTok{(N)]}
\NormalTok{bogosort(data)}
\end{Highlighting}
\end{Shaded}

\begin{verbatim}
([1, 1, 2, 2, 2, 2, 8, 10], 1552)
\end{verbatim}

\hypertarget{exemples-dalgorithmes}{%
\section{Exemples d'algorithmes}\label{exemples-dalgorithmes}}

\hypertarget{fouille-linuxe9aire}{%
\section{Fouille linéaire}\label{fouille-linuxe9aire}}

\hypertarget{bubble-sort}{%
\section{Bubble sort}\label{bubble-sort}}

\hypertarget{insertion-sort}{%
\section{Insertion sort}\label{insertion-sort}}

\hypertarget{binary-search}{%
\section{Binary search}\label{binary-search}}

\hypertarget{heap-sort}{%
\section{Heap sort}\label{heap-sort}}

\hypertarget{complexituxe9-algorithmique}{%
\section{Complexité algorithmique}\label{complexituxe9-algorithmique}}

\bookmarksetup{startatroot}

\hypertarget{thuxe9orie-des-nombres}{%
\chapter{Théorie des nombres}\label{thuxe9orie-des-nombres}}

\hypertarget{arithmuxe9tique-modulaire}{%
\section{Arithmétique modulaire}\label{arithmuxe9tique-modulaire}}

\hypertarget{division-entiuxe8re-1}{%
\subsection{Division entière}\label{division-entiuxe8re-1}}

\hypertarget{congruence-modulo-m}{%
\subsection{\texorpdfstring{Congruence modulo
\(m\)}{Congruence modulo m}}\label{congruence-modulo-m}}

\hypertarget{entiers-et-algorithmes}{%
\section{Entiers et algorithmes}\label{entiers-et-algorithmes}}

\hypertarget{algorithme-dexponentiation-modulaire-efficace}{%
\subsection{Algorithme d'exponentiation modulaire
efficace}\label{algorithme-dexponentiation-modulaire-efficace}}

\hypertarget{nombres-premiers-et-pgcd}{%
\subsection{Nombres premiers et PGCD}\label{nombres-premiers-et-pgcd}}

\hypertarget{algorithme-deuclide-et-thuxe9oruxe8me-de-buxe9zout}{%
\subsection{Algorithme d'Euclide et théorème de
Bézout}\label{algorithme-deuclide-et-thuxe9oruxe8me-de-buxe9zout}}

\hypertarget{inverse-modulo-m}{%
\subsection{\texorpdfstring{Inverse modulo
\(m\)}{Inverse modulo m}}\label{inverse-modulo-m}}

\hypertarget{ruxe9solution-de-congruence}{%
\subsection{Résolution de
congruence}\label{ruxe9solution-de-congruence}}

\hypertarget{petit-thuxe9oruxe8me-de-fermat}{%
\subsection{Petit théorème de
Fermat}\label{petit-thuxe9oruxe8me-de-fermat}}

\hypertarget{cryptographie-uxe0-cluxe9-secruxe8te}{%
\section{Cryptographie à clé
secrète}\label{cryptographie-uxe0-cluxe9-secruxe8te}}

\hypertarget{chiffrement-par-duxe9calage}{%
\subsection{Chiffrement par
décalage}\label{chiffrement-par-duxe9calage}}

\hypertarget{permutation-de-lalphabet}{%
\subsection{Permutation de l'alphabet}\label{permutation-de-lalphabet}}

\hypertarget{masque-jetable}{%
\subsection{Masque jetable}\label{masque-jetable}}

\hypertarget{chiffrement-affine}{%
\subsection{Chiffrement affine}\label{chiffrement-affine}}

\hypertarget{cryptographie-uxe0-cluxe9-publique}{%
\section{Cryptographie à clé
publique}\label{cryptographie-uxe0-cluxe9-publique}}

\hypertarget{chiffrement-rsa}{%
\subsection{Chiffrement RSA}\label{chiffrement-rsa}}

\bookmarksetup{startatroot}

\hypertarget{preuves-et-raisonnement-mathuxe9matique}{%
\chapter{Preuves et raisonnement
mathématique}\label{preuves-et-raisonnement-mathuxe9matique}}

\hypertarget{muxe9thodes-de-preuve}{%
\section{Méthodes de preuve}\label{muxe9thodes-de-preuve}}

\hypertarget{preuve-directe}{%
\subsection{Preuve directe}\label{preuve-directe}}

\leavevmode\vadjust pre{\hypertarget{exm-produit-nombres-pairs-impairs}{}}%
\begin{example}[]\label{exm-produit-nombres-pairs-impairs}

LE PRODUIT DE NOMBRES PAIRS ET IMPAIRS

\end{example}

\leavevmode\vadjust pre{\hypertarget{exm-racines-nombres-pairs}{}}%
\begin{example}[]\label{exm-racines-nombres-pairs}

RACINE DE NOMBRES PAIRS

\end{example}

\leavevmode\vadjust pre{\hypertarget{exm-n2-pair}{}}%
\begin{example}[]\label{exm-n2-pair}

PREUVE QUE \(n^2\) EST PAIR

\end{example}

\leavevmode\vadjust pre{\hypertarget{exm-a-divise-b-divise-c}{}}%
\begin{example}[]\label{exm-a-divise-b-divise-c}

Soit \(a\), \(b\) et \(c\) des entiers. Si \(a|b\) et \(b|c\) alors
\(a|c\).

\end{example}

\hypertarget{preuve-indirecte-par-contraposuxe9e}{%
\subsection{Preuve indirecte (par
contraposée)}\label{preuve-indirecte-par-contraposuxe9e}}

\leavevmode\vadjust pre{\hypertarget{exm-n2-pair-alors-n-pair}{}}%
\begin{example}[]\label{exm-n2-pair-alors-n-pair}

Montrez que si \(n^2\) est pair alors \(n\) est pair.

\end{example}

\leavevmode\vadjust pre{\hypertarget{exm-aplusb-impair-a-b-impair}{}}%
\begin{example}[]\label{exm-aplusb-impair-a-b-impair}

Montrez que si \(a+b\) est impair, alors \(a\) est impair ou \(b\) est
impair.

\end{example}

\leavevmode\vadjust pre{\hypertarget{exm-nombre-premier-impair}{}}%
\begin{example}[]\label{exm-nombre-premier-impair}

Soit \(p\) un nombre premier. Si \(p\neq 2\) alors \(p\) est impair.

\end{example}

\hypertarget{preuve-par-contradiction}{%
\subsection{Preuve par contradiction}\label{preuve-par-contradiction}}

\leavevmode\vadjust pre{\hypertarget{exm-plus-petit-nombre-rationnel}{}}%
\begin{example}[]\label{exm-plus-petit-nombre-rationnel}

EXISTE-T-IL UN PLUS PETIT NOMBRE RATIONNEL POSITIF?

\end{example}

\leavevmode\vadjust pre{\hypertarget{exm-sqrt2-irrationnel}{}}%
\begin{example}[]\label{exm-sqrt2-irrationnel}

PREUVE QUE \(\sqrt{2}\) EST IRRATIONNEL

\end{example}

\leavevmode\vadjust pre{\hypertarget{exm-infinite-nombres-premiers}{}}%
\begin{example}[]\label{exm-infinite-nombres-premiers}

PREUVE QUE QU'IL EXISTE UNE INFINITÉ DE NOMBRES PREMIERS

\end{example}

\leavevmode\vadjust pre{\hypertarget{exm-pas-entiers-equation}{}}%
\begin{example}[]\label{exm-pas-entiers-equation}

Il n'existe pas d'entiers \(x\) et \(y\) tels que \(x^2=4y+2\).

\end{example}

\hypertarget{principe-des-tiroirs-de-dirichlet}{%
\subsection{Principe des tiroirs de
Dirichlet}\label{principe-des-tiroirs-de-dirichlet}}

\leavevmode\vadjust pre{\hypertarget{exm-fonction-hachage}{}}%
\begin{example}[Fonction de hachage]\label{exm-fonction-hachage}

Une fonction de hachage est une fonction qui transforme une suite de
bits de longueur arbitraire en une chaîne de longueur fixe. Du fait
qu'il y a plus de chaînes possibles en entrée qu'en sortie découle par
le principe des tiroirs l'existence de collisions : plusieurs chaînes
distinctes ont le même haché. Rendre ces collisions difficiles à
déterminer efficacement est un enjeu important en cryptographie.

\end{example}

\hypertarget{principe-de-linduction}{%
\section{Principe de l'induction}\label{principe-de-linduction}}

\hypertarget{preuve-par-ruxe9currence}{%
\subsection{Preuve par récurrence}\label{preuve-par-ruxe9currence}}

\leavevmode\vadjust pre{\hypertarget{exm-somme-n-premiers-entiers}{}}%
\begin{example}[]\label{exm-somme-n-premiers-entiers}

PREUVE QUE \(1+2+3+\ldots +n=\frac{n(n+1)}{2}\)

\end{example}

\leavevmode\vadjust pre{\hypertarget{exm-n-plus-petit-2n}{}}%
\begin{example}[]\label{exm-n-plus-petit-2n}

PREUVE QUE \(n<2^n\)

\end{example}

\leavevmode\vadjust pre{\hypertarget{exm-6-divise-7n-1}{}}%
\begin{example}[]\label{exm-6-divise-7n-1}

PREUVE QUE 6 EST UN DIVISEUR DE \(7^n-1\)

\end{example}

\leavevmode\vadjust pre{\hypertarget{exm-space-filling-shapes}{}}%
\begin{example}[]\label{exm-space-filling-shapes}

MONTRER QUE NOUS POUVONS UTILISER DES T-GONES POUR REMPLIR UNE GRILLE
\(2^n \times 2^n\)

\end{example}

\leavevmode\vadjust pre{\hypertarget{exm-exponential-vs-factorial}{}}%
\begin{example}[]\label{exm-exponential-vs-factorial}

MONTRER QUE LA FACTORIELLE CROÎT PLUS RAPIDEMENT QUE L'EXPONENTIELLE

\end{example}

\hypertarget{algorithmes-ruxe9cursifs}{%
\subsection{Algorithmes récursifs}\label{algorithmes-ruxe9cursifs}}

\hypertarget{fonctions-ruxe9cursives}{%
\subsubsection{Fonctions récursives}\label{fonctions-ruxe9cursives}}

\hypertarget{algorithmes-de-type-diviser-pour-ruxe9gner}{%
\subsubsection{Algorithmes de type diviser pour
régner}\label{algorithmes-de-type-diviser-pour-ruxe9gner}}

\bookmarksetup{startatroot}

\hypertarget{duxe9nombrement}{%
\chapter{Dénombrement}\label{duxe9nombrement}}

\hypertarget{notions-de-base}{%
\section{Notions de base}\label{notions-de-base}}

\hypertarget{principe-des-nids-de-pigeon-principe-des-tiroirs-de-dirichlet}{%
\section{Principe des nids de pigeon (principe des tiroirs de
Dirichlet)}\label{principe-des-nids-de-pigeon-principe-des-tiroirs-de-dirichlet}}

\hypertarget{permutations-et-combinaisons}{%
\section{Permutations et
combinaisons}\label{permutations-et-combinaisons}}

\hypertarget{relations-de-ruxe9currence-et-duxe9nombrement}{%
\section{Relations de récurrence et
dénombrement}\label{relations-de-ruxe9currence-et-duxe9nombrement}}

\bookmarksetup{startatroot}

\hypertarget{graphes}{%
\chapter{Graphes}\label{graphes}}

\hypertarget{terminologie-et-types-de-graphes}{%
\section{Terminologie et types de
graphes}\label{terminologie-et-types-de-graphes}}

\hypertarget{repruxe9sentation-des-graphes}{%
\section{Représentation des
graphes}\label{repruxe9sentation-des-graphes}}

\hypertarget{repruxe9sentation-par-listes-dadjacence}{%
\subsection{Représentation par listes
d'adjacence}\label{repruxe9sentation-par-listes-dadjacence}}

\hypertarget{repruxe9sentation-par-matrice-dadjacence}{%
\subsection{Représentation par matrice
d'adjacence}\label{repruxe9sentation-par-matrice-dadjacence}}

\hypertarget{chemins-dans-un-graphe}{%
\section{Chemins dans un graphe}\label{chemins-dans-un-graphe}}

\hypertarget{chemins-circuits-cycles}{%
\subsection{Chemins, circuits, cycles}\label{chemins-circuits-cycles}}

\hypertarget{duxe9nombrement-de-chemins}{%
\subsection{Dénombrement de chemins}\label{duxe9nombrement-de-chemins}}

\hypertarget{chemins-et-circuits-euluxe9riens}{%
\subsection{Chemins et circuits
eulériens}\label{chemins-et-circuits-euluxe9riens}}

\hypertarget{chemins-et-circuits-hamiltoniens}{%
\subsection{Chemins et circuits
hamiltoniens}\label{chemins-et-circuits-hamiltoniens}}

\hypertarget{probluxe8me-du-plus-court-chemin}{%
\section{Problème du plus court
chemin}\label{probluxe8me-du-plus-court-chemin}}

\bookmarksetup{startatroot}

\hypertarget{arbres}{%
\chapter{Arbres}\label{arbres}}

\hypertarget{introduction-aux-arbres}{%
\section{Introduction aux arbres}\label{introduction-aux-arbres}}

\hypertarget{applications-des-arbres}{%
\section{Applications des arbres}\label{applications-des-arbres}}

\hypertarget{parcours-dun-arbre}{%
\section{Parcours d'un arbre}\label{parcours-dun-arbre}}

\hypertarget{arbres-et-tri}{%
\section{Arbres et tri}\label{arbres-et-tri}}

\hypertarget{arbres-et-recouvrement}{%
\section{Arbres et recouvrement}\label{arbres-et-recouvrement}}

\hypertarget{arbres-guxe9nuxe9rateurs-de-couxfbt-minimal}{%
\section{Arbres générateurs de coût
minimal}\label{arbres-guxe9nuxe9rateurs-de-couxfbt-minimal}}

\bookmarksetup{startatroot}

\hypertarget{ruxe9fuxe9rences}{%
\chapter*{Références}\label{ruxe9fuxe9rences}}
\addcontentsline{toc}{chapter}{Références}

\markboth{Références}{Références}

\hypertarget{refs}{}
\begin{CSLReferences}{0}{0}
\end{CSLReferences}


\backmatter

\end{document}
